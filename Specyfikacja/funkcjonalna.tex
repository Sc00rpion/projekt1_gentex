\documentclass[a4paper,11pt]{article}
\usepackage{latexsym}
\usepackage[MeX]{polski}
\usepackage[english,polish]{babel}
\usepackage{polski}
\usepackage[utf8]{inputenc}

\usepackage{amsmath}
\usepackage{txfonts,qtimes,qtxmath}

\setlength{\headheight}{0pt}
\setlength{\textheight}{23.5cm}
\setlength{\textwidth}{15.92cm}
\setlength{\footnotesep}{5mm}
\setlength{\footskip}{10mm}
\setlength{\oddsidemargin}{0mm}
\setlength{\evensidemargin}{0mm}
\setlength{\topmargin}{0mm}
\setlength{\headsep}{5mm}
\setlength{\parindent}{0mm}
\setlength{\parskip}{2.5mm}


\title{ \Huge {\textbf{ Specyfikacja funkcjonalna}}}
\author{Maciej Skarbek \\Nr albumu:271088}

\frenchspacing

\begin{document}
\maketitle

\begin{itemize}
\item
	\texttt{
		\pmb{NAZWA}
			\begin{quote}
			\begin{quote}
			\begin{quote}
						gentex - Generator tekstu 		
			\end{quote}
			\end{quote}
			\end{quote}
			}
\item
	\texttt{
		\pmb{OPIS OGOLNY}
			\begin{quote}
			\begin{quote}
			\begin{quote}
			Program gentexr - Program ma za zadanie generować tekst wyjściowy(formatu txt) na podstawie analizy innych tekstów(formatu txt) wykorzystując przy tym łańcuchy Markova oraz N-gramy. Aplikacja jest programem konsolowym urochomianym przez podanie odpowiednich parametrów wywołania. Możliwe jest generowanie tekstu o zadanej ilosc słow,akapitów,o zdanym rozmierze oraz wykorzystywanego stopnia N-gramu przy generowaniu tekstu. Program działa w jednym z dwoch trybow. Pierwszy tryb to generowanie tekstu na podstawie 'czystych tekstow'(np:ksiażki,artykuły) oraz tryb generowania tekstu na podstawie pliku posredniego.
			\end{quote}
			\end{quote}
			\end{quote}
			}
\item
	\texttt{
		\pmb{WYWOŁANIE}
			\begin{quote}
			\begin{quote}
			\begin{quote}
				gentex [opcje] plik\_bazowy1.txt plik\_bazowy2.txt ...
			\end{quote}
			\end{quote}
			\end{quote}
			}
\item
\texttt{
	\pmb{OPCJE}
	\begin{quote}
		\begin{quote}
			\begin{quote}
			\pmb{\large{ -o nazwa\_pliku}}		Nazwa pliku gdzie ma byc zapisany wygenerowany tekst\\\\
			\pmb{\large{-s liczba\_słów	}}	Liczba słow z jakiej ma sie skladac wygenerowany tekst\\\\
			\pmb{\large{-a liczba\_akapitów}}	Liczba akapitów z której składac ma sie wygenerowany tekst\\\\
			\pmb{\large{-n rząd\_n-gramów}}	Rząd n-gramów w oparciu o który będzie generowany tekst\\\\
			\pmb{\large{-stat nazwa\_pliku}}	Generowanie statystyk tekstu bazowego i wynikowego zapisana do pliku\\\\
			\pmb{\large{-pout nazwa\_pliku}} Generuje plik posredni w formie binarnej z ktorego mozna uzyc do generowania tekstow\\\\
			\pmb{\large{-pin nazwa\_pliku}} Generowanie tekstu wynikowego z pliku posredniego
			\end{quote}
		\end{quote}
	\end{quote}
	}
\item
\texttt{
	\pmb{UWAGI}
	\begin{quote}
		\begin{quote}
			\begin{quote}
				Plik posredni jest generowany w formie bliku binarnego co ma na celu bezawaryjne i szybkie działanie programu. 		
			\end{quote}
		\end{quote}
	\end{quote}
}	
\end{itemize}
\end{document}
